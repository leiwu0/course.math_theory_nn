\documentclass[12pt]{article}
\usepackage[english]{babel}
\usepackage[utf8x]{inputenc}
\usepackage[T1]{fontenc}
\usepackage{scribe}
\usepackage{listings}

\Scribe{Shuhai Zhao, Yilei Han}
\Lecturer{Lei Wu}
\LectureNumber{1}
\LectureDate{July 2}
\LectureTitle{An Introduction to Supervised Learning}

\lstset{style=mystyle}

\begin{document}
    \MakeScribeTop

\section{Supervised Learning}
Some basic terminologies:
\begin{itemize}
\item \textit{features}:  The set of attibutes, often represented as a vector, associated to an example.
\item \textit{Hypothesis space}: A set $\mathcal{F}$ of functions mapping features to the set of labels $\mathcal{Y}$.
\item \textit{Loss function}: A function $l$ that measures the difference, or loss, between a predicted label and a true label: $l:\mathcal{Y}×\mathcal{Y}\to \mathbb{R}_{+}$, for example, $l(y,y')=(y-y')^{2}$.
\end{itemize}


\bibliographystyle{abbrv}           % if you need a bibliography
\bibliography{mybib}                % assuming yours is named mybib.bib

\end{document}